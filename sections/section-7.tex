\section{SUMMARY AND CONCLUSIONS 总结和结论}

\begin{paracol}{2}
    
    This paper considers the undrained shear strength of saturated "undisturbed" clays as determined by laboratory tests. For most normally consolidated clays, the value of undrained shear strength, $S_u$, measured with UU tests, is only 40 to 80 percent of that measured with CIU tests having a consolidation pressure equal to the in situ vertical effective stress. A portion of this difference is attributed to the fact that sampling necessarily involves the release of in situ shear stresses, since $K_0$ is significantly less than unity. In addition, further disturbance due to tube sampling, extrusion, trimming, etc., generally reduces the actual effective stress in soil specimens to a value far below that existing in the ground.

    \switchcolumn

    本文考虑了通过实验室试验确定的饱和“未扰动”黏土的不排水剪切强度。 对于大多数正常固结的黏土,UU试验测得的不排水抗剪强度$S_u$值仅为固结压力等于现场垂直有效应力的CIU试验测得的不排水抗剪强度s的40$\%$至80$\%$。 这种差异的一部分归因于这样一个事实,即采样必然涉及就地剪切应力的释放,因为$K_0$明显小于1。 此外,由于试管采样,挤压,修整等引起的进一步干扰通常会将土壤样本中的实际有效应力减小到远低于地面存在的值。

    \switchcolumn*

    An estimate of the isotropic effective stress in a specimen following perfect sampling, in which only the in-situ shear stresses are released, can be obtained by reconsolidating a specimen to the $K_0$ condition and then releasing the shear stresses at constant mass. A comparison of this theoretical stress, termed $\overline{\sigma}_{ps}$, with the effective stress $\overline{\sigma}_r$ as actually measured in laboratory specimens is used to indicate the degree of sample disturbance. Test data on tube samples of several moderately sensitive clays show average values of 2.8 to 5 for the ratio $\overline{\sigma}_{ps}/\overline{\sigma}_r$.

    \switchcolumn

    通过将样品重新固结为$K_0$条件,然后以恒定质量释放剪切应力,可以得出理想采样后各向同性有效应力的估算值,其中仅释放原位剪切应力。 将这个称为$\overline{\sigma}_{ps}$的理论应力与实际在实验室样品中测得的有效应力$\overline{\sigma}_r$进行比较,以表明样品扰动的程度。 几种中度敏感黏土的试管样品的试验数据显示,比率$\overline{\sigma}_{ps}/\overline{\sigma}_r$的平均值为2.8至5。

    \switchcolumn*

    It is felt that UU strengths on tube samples are often significantly below those for truly undisturbed samples and that some of the reported agreements between such strengths and those backfigured from field observations may well have involved compensating errors. Furthermore, values of s,, ~3, and A I measured from CIU tests on tube samples, which generally exhibit significant volume decreases during reconsolidation in the laboratory, are liable to errors on the unsafe side; s, and ~ being too large and AI being too low.

    \switchcolumn

    有人认为,试管样品上的UU强度通常远低于真正不受干扰的样品,一些报告对这种强度与现场观察得出的一些UU强度很可能涉及抵消误差达成共识。此外,通过CIU试验对试管样品测得的$S_u$,$\overline{\phi}$和$A_f$值通常会在实验室中的固结过程中显示出明显的体积减小,这在不安全时容易出错;\footnote{
        Need to change.
    }$S_u$和$\overline{\phi}$太大而$A_f$太低。

    \switchcolumn*

    Methods are proposed for adjusting the values of $S_u$, measured from UU and CU tests to that corresponding to an undrained triaxial compression test on a specimen after perfect sampling. An examination of UU test results suggests that such specimens behave as overconsolidated specimens. Based on this concept, the authors suggest that the ratio ​​$\overline{\sigma}_{ps}/\overline{\sigma}_r$ be considered as an overconsolidation ratio and the UU strengths corrected accordingly.

    \switchcolumn

    提出了将UU和CU试验中测得的$S_u$值调整为与完美采样后样本的不排水三轴压缩试验相对应的$S_u$值的方法。 对UU试验结果的检查表明,此类样本表现为超固结样本。 基于此概念,作者建议将比率​​$\overline{\sigma}_{ps}/\overline{\sigma}_r$视为超固结比,并相应地校正UU强度。

    \switchcolumn*

    The paper proposes that the volume decrease attendant with the decrease in effective stresses during sampling and their reapplication during reconsolidation in the laboratory prior to a CIU test can be accounted for, with many clays, by correcting the results of the CIU test through Hvorslev parameters. An alternate approach for obtaining an adjusted $S_u$ from CU tests employs test data on specimens consolidated to pressures much greater than the in silu overburden pressure.

    \switchcolumn

    本文提出,对于许多黏土,可以通过Hvorslev参数校正CIU试验的结果,来解释伴随着CIU试验在实验室中取样时有效应力的减少以及在固结过程中重新施加应力时体积的减少。从CU试验中获得调整后的$S_u$的另一种方法是,将样本固结到比井上覆盖层压力大得多的压力的试验数据。

    \switchcolumn*

    Two case studies are presented in which the results of CIU and UU tests, which had showed a 50 per cent discrepancy in strength, are analyzed in accordance with the procedures proposed in this paper. Following correction of the test data for sample disturbance, there was good agreement (within 5 to 10 percent) between the strengths obtained from UU and CIU tests.

    \switchcolumn

    提出了两个案例研究,其中CIU和UU试验的结果已按照本文提出的程序进行了分析,结果表明强度存在50$\%$的差异。 校正了样品干扰的试验数据后,UU和CIU试验获得的强度之间有很好的一致性(在5$\%$到10$\%$之内)。

    \switchcolumn*

    It is suggested that values of residual effective stress, $\overline{\sigma}_r$, be determined on representative "undisturbed" samples for all important jobs as a quantitative measure of the amount of disturbance caused by sampling. It is further hoped that the methods proposed here will be investigated for other clays and that the undrained shear strength corresponding to perfect sampling can eventually be related in a rational manner to actual field strengths.

    \switchcolumn

    建议在所有重要工作的代表性“不受扰动”样本上确定残余有效应力$\overline{\sigma}_r$的值,以定量衡量由样本引起的干扰。 进一步希望这里提出的方法将用于其他黏土,并且希望与理想采样相对应的不排水剪切强度最终可以合理地与实际场强相关联。

\end{paracol}

\Paragraph{Acknowledgments: 致谢:}

\begin{paracol}{2}

    Mr. W. A. Bailey, Mr. L. G. Bromwell and Mr. Paulo da Cruz, present or former research assistants in soil mechanics assisted by performing most of the tests described in this paper. Several of the authors' colleagues reviewed a draft of the paper and made many helpful suggestions; Professors R. V. Whitman, C. W. Lovell, and H. M. Horn and Mr. Bailey were particularly beneficial in this respect.

    \switchcolumn

    Mr. W. A. Bailey, Mr. L. G. Bromwell and Mr. Paulo da Cruz,现任或前任岩土力学研究助理通过执行本文所述的大多数试验得到了帮助。 作者的几个同事审阅了本文的草案,并提出了许多有益的建议。 在这方面,Professors R. V. Whitman, C. W. Lovell, and H. M. Horn and Mr. Bailey 提供的建议特别有帮助。

    \switchcolumn*

    Most of the tests on Lagunillas clay and the Kawasaki clays were conducted in conjunction with engineering projects for the Creole Petroleum Corporation and the Esso Research and Engineering Co., respectively. Their permission to publish these results is appreciated.

    \switchcolumn
       
    对Lagunillas黏土和Kawasaki黏土的大多数试验分别是与Creole石油公司和Esso Research and Engineering Co.的工程项目一起进行的。 感谢他们发布这些结果的许可。

    \switchcolumn*

    Laboratory tests and certain theoretical analyses described in this paper were done in conjunction with research in the M.I.T. Soils Laboratory, Department of Civil Engineering, and were sponsored in part by the U.S. Army Corps of Engineers, Waterways Experiment Station, Vicksburg, Miss.

    \switchcolumn
       
    本文所述的实验室试验和某些理论分析是结合M.I.T.土木工程系岩土实验室,部分由美国陆军工程兵团,密西西比州维克斯堡的水路实验站赞助。
    
\end{paracol}