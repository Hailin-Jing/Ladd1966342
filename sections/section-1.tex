\begin{paracol}{2}
    
    This paper considers the determination of undrained shear strength, su of saturated "undisturbed" samples of clay from two types of laboratory tests, namely (1) one in which the soil specimen is not consolidated prior to testing but sheared at the existing moisture content, that is, unconsolidated-undrained test (UU), and (2) one in which the soil specimen is consolidated prior to testing, that is, consolidated-undrained test (CU). A comparison of results from the two types of tests is given, and methods of adjusting the results from each, to put them on an equal basis, are described.

    \switchcolumn

    本文考虑了从两种类型的实验室试验中确定不饱和黏土“不排水”样品的不排水剪切强度,即(1)一种在试验前土体样品未固结而是在现有含水量下剪切的土体样品,  (2)在试验前将土体样本固结的一种方法,即固结排水试验(CU)。 给出了两种试验结果的比较,并描述了调整每种试验结果以使其相等的方法。

    \switchcolumn*

    A determination of the factor of safety against a shear failure of a foundation, embankment, natural slope, retaining wall, etc., must usually consider a failure, in which the soil does not undergo Consolidation during shear. The undrained shear strength is the soil parameter needed for this determination. The selection of the proper value of this soil parameter for a clay can be the step in the investigation which is the most difficult and the one which is most subject to large error. 

    \switchcolumn
   
    确定地基,路堤,自然坡度,挡土墙等的抗剪破坏的安全系数通常必须考虑一种破坏,即在剪切过程中土体不会发生固结。 不排水的剪切强度是该确定所需的土体参数。 为黏土选择该土体参数的适当值可能是研究中最困难的步骤,也是最大误差较大的步骤。

    \switchcolumn*

    To obtain the undrained shear strength of an element of clay in a field problem, the engineer would like to test in the laboratory a specimen of clay having the same moisture content and the same effective stress system that exist in the field element. Considerable experience has shown that, unfortunately, both the moisture content and the effective stresses on a field element cannot be duplicated simultaneously on a laboratory specimen. The engineer must therefore choose between the following approaches: (1) keep the moisture content of the laboratory specimen equal to that desired and run an undrained test (UU), or (2) make the effective stresses on the laboratory specimen equal to those desired and run an undrained test (CU). Because the moisture content and initial effective stress for the two tests are unequal, the strengths measured by the two tests are normally different. This paper presents test data from a wide variety of clays which show that strength values from UU tests are only 40 to 97 percent of the values from CU tests.

    \switchcolumn

    为了获得在现场问题中黏土元素的不排水剪切强度,工程师希望在实验室中试验具有与现场元素相同的水分含量和相同的有效应力系统的黏土样品。 相当多的经验表明,不幸的是,水分和现场土体上的有效应力无法同时复制到实验室样本上。 因此,工程师必须在以下方法之间进行选择:(1)保持实验室样品的水分含量与期望值相等,并进行不排水试验(UU),或者(2)使实验室样品上的有效应力等于期望值 并进行不排水试验(CU)。 由于两次试验的水分含量和初始有效应力不相等,因此两次试验测得的强度通常不同。本文介绍了各种黏土的试验数据,这些数据表明,UU试验的强度值仅是CU试验值的40$\%$至97$\%$。

    \switchcolumn*

    The essential cause of the usually large and significant difference between UU and CU test results is the change in soil structure which occurs during the process of removing a chunk of soil from the ground, transporting it to the laboratory, trimming a test specimen, and mounting the specimen in the equipment for shearing. The next section of this paper considers the changes in soil structure as evidenced by changes in effective stress resulting from this entire process, termed here sampling. 

    \switchcolumn

    UU和CU试验结果之间通常存在较大差异的根本原因是土体结构的变化,这种变化发生在从地面上去除一块土体,将其运送到实验室,修剪测试样本,并将样本安装在剪切设备中。本文的下一部分考虑了土体结构的变化,这一过程由整个过程(在此称为采样)导致的有效应力变化得到了证明。

    \switchcolumn*
    
    Methods are proposed for adjusting values of $s_u$ measured from UU and CU tests to that corresponding to an undrained triaxial compression test on a specimen after perfect sampling, for which only the in situ shear stresses were released. It is emphasized that clays with a very sensitive structure, such as the quick clays and clays with a significant amount of natural cementation, are excluded from consideration. Furthermore it is not proposed that the adjusted value of su, which only corresponds to triaxial compression at normal strain rates, is necessarily the appropriate strength for a phi = 0 analysis, since the effects of the intermediate principal stress, the possible reorientation of principal planes, different rates of strain, etc. are not taken into consideration. However, the methods suggested for evaluating the effects of sampling should enable the development, for many types of clay, of a more rational approach than now exists for predicting the strength in the field. 

    \switchcolumn
    
    本文提出了将UU和CU试验中测得的$s_u$值调整为与完美采样后对试样进行不排水三轴压缩试验相对应的方法,为此仅释放了原位剪切应力。 应该强调的是,不考虑具有非常敏感结构的黏土,例如速成黏土和具有大量天然胶结的黏土。 此外,由于中间主应力的影响,主平面可能的重新定向,因此不建议将$s_u$的调整值仅对应于法向应变率下的三轴压缩,它不一定是$\phi=0$分析的适当强度,不同的应变率等未考虑在内。 但是,对于许多类型的黏土,建议的评估取样效果的方法应该能够开发出比现在更合理的方法来预测场地土强度。

\end{paracol}