\begin{table}[!htb]
    \centering
    \caption{PREDICTION OF UNDRAINED STRENGTH AT IN SITU VOID RATIO FROM CIU TRIAXIAL TESTS BY PREVIOUS METHODS FOR KAWASAKI CLAY I.}
    \addtocounter{table}{-1}
    \vspace{-8pt}
    \renewcommand{\tablename}{表}
    \caption{川崎黏土I的先前方法从CIU三轴试验预测原位空隙率的不排水强度。}
    \vspace{4pt}
    \renewcommand{\tablename}{Table}
    \begin{threeparttable}[b]
        \begin{tabularx}{\textwidth}{ll}
            \toprule
            Method & $S_u{\rm (kg/cm^2)~for}~ \Delta{}e/(1+e_0)=0$\\
            \midrule
            \citet{Schmertmann1956940} & ~~0.3 to 0.45 \\
            \citet{Calhoon1956925} & $\sim$0.85 \\
            \citet{Casagrande1947} & $\sim$0.5 \\
            Average of unconfined\tnote{a} compression tests & ~~0.45\tnote{b} \\
            Average of $\overline{UU}$ and top one third of unconfined compression tests & ~~0.5\tnote{b} \\
            \bottomrule
        \end{tabularx}%
        \begin{tablenotes}
            \item[a] Many of the specimens contained lenses of sand, silt, or shells which caused very low unconfined strengths. 许多标本包含沙粒,粉尘或贝壳状的透镜,这些透镜的强度很低。
            \item[b] Corrected to correspond to $t_f=5$ hr. 校正为对应于 $t_f=5$小时。
        \end{tablenotes}
    \end{threeparttable}
    \label{table:3}%
\end{table}
